\documentclass[12pt, a4paper]{article}

\usepackage[brazilian]{babel}
\usepackage[T1]{fontenc}
\usepackage[utf8]{inputenc}

\usepackage[section]{minted}
\usepackage{amsthm}

\usepackage[colorlinks=true]{hyperref}
\usepackage{color}

\title{Lista - Algoritmos e Estrutura de Dados}
\author{J Carlos Viana F}
\date{}

\definecolor{bg}{rgb}{0.95,0.95,0.95}

\newmintedfile[ccode]{c}{
bgcolor=bg,
fontfamily=tt,
linenos=true,
numberblanklines=true,
numbersep=12pt,
numbersep=5pt,
gobble=0,
frame=leftline,
framerule=0.4pt,
framesep=2mm,
funcnamehighlighting=true,
tabsize=4,
obeytabs=false,
mathescape=false
samepage=false, %with this setting you can force the list to appear on the same page
showspaces=false,
showtabs =false,
texcl=false,
}

\newmintedfile[haskellcode]{haskell}{
bgcolor=bg,
fontfamily=tt,
linenos=true,
numberblanklines=true,
numbersep=12pt,
numbersep=5pt,
gobble=0,
frame=leftline,
framerule=0.4pt,
framesep=2mm,
funcnamehighlighting=true,
tabsize=4,
obeytabs=false,
mathescape=false
samepage=false, %with this setting you can force the list to appear on the same page
showspaces=false,
showtabs =false,
texcl=false,
}

\begin{document}

% título
\makeatletter
\begin{center}

\texttt{\Large \@title}
\end{center}

\emph{Aluno: \@author.}
\makeatother

% questões

\begin{enumerate}

\item (resposta)


$f(n) = 2n^{2}+3n+4$ é $\mathcal{O}(n^{2})$; logo $g(n)=n^{2}$

$2n^{2}+3n+4 \leq cn^{2}$\\
$2n^{2}+3n+4 \leq 4n^{2}$

$f(n)=\mathcal{O}(n^{2})$,\\
$c=4$ e $n_{0}=3$

\item (resposta)

$f(n)=n^{3}$ é $\mathcal{O}(n^{2})$; $g(n)=n^{2}$

$n^{3} \leq cn^{2}$

não há $c$ e $n_{0}$ tal que $n^{3} \leq cn^{2}$\\
Logo $f(n)$ não é $\mathcal{O}(n^{2})$

\item (resposta)

$f(n)=2^{n+1}$ é $\mathcal{O}(2^{n})$; $g(n)=2^{n}$

$2^{n+1} \leq c2^{n}$\\
$2^{n+1} \leq 2.2^{n}$\\
$2^{n+1} \leq 2^{n+1}$

$f(n)=\mathcal{O}(2^{n})$,\\
$c=2$ e $n_{0}=1$

\item (resposta)

$T(n)=1+T(n-1)$
$=1+1+T(n-2)$
$=2+T(n-2)$
$=2+1+T(n-3)$
$=3+T(n-3)$

$T(n)=k+T(n-k)$\\
para $k=n-1$ temos\\
$T(n)=n-1+T(1)$\\
Logo, a busca linear/sequencial é $\mathcal{O}(n)$

\item (resposta)

$T(n)=1+T(\frac{n}{2})=1+1+T(\frac{n}{4})$
$=2+T(\frac{n}{4})=2+1+T(\frac{n}{8})$
$=3+T(\frac{n}{8})$

$T(n)=k+T(\frac{n}{2^{k}})$\\
$2^{k}=n \Rightarrow \log_{2}2^{k}=\log_{2}n \Rightarrow k=\log_{2}n$\\
$T(n)=\log_{2}n+T(1)$\\
Logo, a busca binária é $\mathcal{O}(\log n)$

\item (resposta)

$T(n)=n+2T(\frac{n}{2})=n+2[\frac{n}{2}+2T(\frac{n}{4})]$
$=n+n+4T(\frac{n}{4})=2n+4T(\frac{n}{4})=2n+4[\frac{n}{4}+2T(\frac{n}{8})]$
$=3n+8T(\frac{n}{8})$

$T(n)=kn+2^{k}T(\frac{n}{2^{k}})$\\
$2^{k}=n \Rightarrow \log_{2}2^{k}=\log_{2}n \Rightarrow k=\log_{2}n$\\
$T(n)=n\log_{2}n + nT(1)$
De modo que o \textit{quicksort} é $\mathcal{O}(n\log n)$

\item (resposta)

$f(n)+g(n)=\mathcal{O}(max\{f(n),g(n)\})$

$f(n) \leq c_{1}h(n)$\\
$g(n) \leq c_{2}h(n)$

$f(n)+g(n) \leq c_{1}h(n)+c_{2}h(n)$\\
$f(n)+g(n) \leq (c_{1}+c_{2})h(n)$\\
$f(n)+g(n) \leq d.h(n)$, sendo $d = c_{1}+c_{2}$\\
$f(n)+g(n) \leq d.\mathcal{O}(max\{f(n),g(n)\})$

$f(n)+g(n)=\mathcal{O}(max\{f(n),g(n)\})$ para $d = c_{1}+c_{2}$ e um dado $n_{0}$

\item (resposta)

Para o \textit{merge}:

\ccode{cod/08.1.1.c}
\ccode[firstnumber=20]{cod/08.1.2.c}

Para o \textit{mergesort} (não é alterado):

\ccode[firstnumber=33]{cod/08.2.c}

\item (resposta)

\ccode{cod/09.1.c}
\ccode[firstnumber=9]{cod/09.2.c}

\item (resposta)

\haskellcode{cod/10.hs}

\item (resposta)

\haskellcode{cod/11.hs}

\item (resposta)

\haskellcode{cod/12.hs}

\item (resposta)

\haskellcode{cod/13.hs}



\end{enumerate}


\end{document}