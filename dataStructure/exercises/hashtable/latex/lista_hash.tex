\documentclass[a4paper,11pt]{article}

\usepackage[brazilian]{babel}
\usepackage[T1]{fontenc}
\usepackage[utf8]{inputenc}

\usepackage[colorlinks=true]{hyperref}
\usepackage{color}

\usepackage[left=2.5cm,right=3cm,top=2.5cm,bottom=3cm]{geometry}
\setlength{\parindent}{3em}
\setlength{\parskip}{1em}
\renewcommand{\baselinestretch}{1.1}
\usepackage{indentfirst}

\usepackage{minted}

\definecolor{bg}{rgb}{0.95,0.95,0.95}

\newmintedfile[ccode]{c}{
bgcolor=bg,
fontfamily=tt,
linenos=true,
numberblanklines=true,
numbersep=12pt,
numbersep=5pt,
gobble=0,
frame=leftline,
framerule=0.4pt,
framesep=2mm,
funcnamehighlighting=true,
tabsize=4,
obeytabs=false,
mathescape=false
samepage=false, %with this setting you can force the list to appear on the same page
showspaces=false,
showtabs =false,
texcl=false,
}

\begin{document}

\begin{flushright}
\makeatletter
\textit{\@date}
\makeatother
\end{flushright}

\begin{center}
{\Large \textbf{Exercício de Hashtable}}
\end{center}

\textit{J Carlos Viana Filho}


\begin{enumerate}

\item (resposta)

\ccode{code/01.c}

\item (resposta)

\ccode[lastline=13]{code/02.c}
\ccode[firstline=14,firstnumber=14,lastline=57]{code/02.c}
\ccode[firstline=58,firstnumber=58,lastline=100]{code/02.c}
\ccode[firstline=101,firstnumber=101,lastline=143]{code/02.c}
\ccode[firstline=144,firstnumber=144,lastline=185]{code/02.c}
\ccode[firstline=186,firstnumber=186,lastline=230]{code/02.c}
\ccode[firstline=231,firstnumber=231]{code/02.c}

\item (resposta)

Usando o método \textit{open addressing}, 61 estará localizado no índice 0 da table, 62 no índice 1, 63 no índice 2, 64 no índice 3 e 65 no índice 4.

Usando o método \textit{close addressing}, esses valores estarão no índice 0, que apontará para uma lista encadeada com nós que contém os seguintes valores: 61, 62, 63, 64 e 65, nesta ordem.

\end{enumerate}

\end{document}